Title: The New Muon g-2 Experiment: E989 Status and Progress Update

The Higgs Boson has been discovered at the LHC. With the final piece of the Standard Model locked in place it is time to focus on the tensions around the edges. Work in the neutrino and cosmic sectors indicate a need for new physics at higher energy scales. A key tool in that search is the precision measurement of Standard Model predictions. Measurement of the muon anomalous magnetic moment ($a_{\mu}$) has long proven to be a useful guide. The measurement can be conducted with exceptional sensitivity using the technique first developed at CERN and most recently implemented at BNL where the E821 experiment measured $a_{\mu}$ to 540$\,$ppb. This measurement resulted in a discrepancy from the Standard Model at a level greater than 3$\sigma$. The New Muon g-2 experiment at Fermilab (E989) aims to improve this precision by a factor of 4, down to 140$\,$ppb. Coupled with ongoing improvements in the theoretical calculation the new result will yield vital clues in the search for new physics. 
The E989 collaboration is making great progress and we will discuss: the completion of moving the superconducting electromagnet from Brookhaven National Lab to Fermilab, on site construction at Fermilab, and detector development focusing on PbF$_{2}$ calorimetry with SiPM readout.
