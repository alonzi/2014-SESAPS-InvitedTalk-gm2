Title: The New Muon g-2 Experiment: E989 Status and Progress Update

The Higgs Boson has been discovered at the LHC, locking in the final piece of the Standard Model. Now it is time to focus on the tensions around the edges of the model. Work in the neutrino and cosmic sectors indicate a need for new physics at higher energy scales. A key tool in that search is the precision measurement of Standard Model predictions. Measurement of the muon anomalous magnetic moment ($a_{\mu}$) has long proved to be a useful guide, due in part to the exceptional sensitivity of the measurement technique first developed at CERN and most recently implemented at BNL where the E821 experiment measured it to 540$\,$ppb. This measurement resulted in a discrepancy from the Standard Model at a level greater than 3$\sigma$. The New Muon g-2 experiment at Fermilab (E989) aims to improve this precision by a factor of 4, down to 140$\,$ppb. Coupled with ongoing improvements in the theoretical calculation the new result will yield vital clues. The collaboration is making gr
 eat progress and we will discuss: the completion of moving the superconducting electromagnet from Brookhaven National Lab to Fermilab, on site construction at Fermilab, detector development focusing on PbF$_{2}$ calorimetry with SiPM readout, and precision magnetic field monitors based on NMR technology.
